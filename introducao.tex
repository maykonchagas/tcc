% ----------------------------------------------------------------------- %
% Arquivo: introducao.tex
% ----------------------------------------------------------------------- %
\chapter{Introdução}
\label{c_introducao}

A disponibilidade de serviços e aplicações acessíveis remotamente na Internet se tornou um processo relativamente simples de implementação. O avanço das tecnologias de redes de computadores foi responsável pela construção dessas aplicações e a facilidade para acesso às mesmas. No entanto, além de manter a própria aplicação, administradores de sistemas necessitam ainda manter uma base de usuários própria com informações e níveis de privilégio para permitir acesso às aplicações, tornando o trabalho custoso \cite{moreira:11}. 

Do lado do usuário, com tantos serviços disponíveis, é permitida a criação de múltiplas identidades para acesso a esses serviços. Cada novo serviço que o usuário deseja acessar, este deve repassar algumas informações pessoais e um nome de usuário e senha para acessar o serviço. Criar um nome de usuário e senha para cada serviço seria uma boa prática de segurança, porém, administrar essas informações é uma tarefa difícil para os usuários, diante da grande gama de serviços que são oferecidos na Internet \cite{wangham:10a}.
	
Segundo \cite{kallela:08, wangham:10b}, o problema de gestão de identidades afeta tanto o usuário, que repete informações sem dar a devida importância ou usa senhas fracas, quanto as empresas que além de prover o serviço ainda precisam se preocupar com a gestão de identidades dos usuários, gerando custos administrativos e de infraestrutura. O conceito de gestão de identidades federadas surgiu como uma opção de solução para estes problemas.

No modelo de gestão de identidades federadas \cite{josan:05, pope:05, bhargav:07}, objetiva-se remover a complexidade do usuário em ter que administrar um nome de usuário e senha para cada serviço que deseja acessar. O conceito de federação visa minimizar as demandas dos provedores de serviço e de usuários de um domínio, delegando serviços bem específicos para cada elemento da estrutura da federação. Uma federação é composta por dois componentes principais, (1) provedores de identidades, \ac{IdP}, responsáveis pela autenticação e gerenciamento das informações dos usuários de um domínio, além de definir o método de autenticação o IdP deve garantir que cada usuário do seu domínio tenha um identificador único, e (2) provedores de serviços, \ac{SP}, que disponibilizam serviços para acesso dos usuários, podem requisitar informações adicionais para garantir acesso a um determinado serviço, independente do domínio \cite{moreira:11}.
	
Neste modelo, informações dos usuários são compartilhadas entre provedores de identidade e provedores de serviços, que possuem relações de confiança entre si e pertencem ao círculo de confiança da federação. Este modelo provê a facilidade de autenticação única, \ac{SSO}, que garante ao usuário passar uma única vez pelo processo de autenticação e acessar qualquer provedor de serviços da federação, cabendo a estes provedores realizarem somente o controle de acesso dos usuários. O modelo de gestão de identidades federadas se mostra vantajoso para o usuário, que necessitará de uma única identidade para acessar os diversos serviços da federação, e para o administrador do sistema, que ao prover um serviço para a federação não precisará se preocupar com a autenticação e com o cadastro de usuários.
	
Para realizar o gerenciamento de identidades federadas existem diferentes soluções, estas podem ser: baseadas em padrões abertos desenvolvidos por grandes consórcios; soluções proprietárias e projetos abertos. Muitas  soluções incluem funcionalidades similares mas se diferenciam no escopo da solução e na aplicabilidade para diferentes cenários. Três grandes implementações se destacam: a Liberty Alliance\footnote{http://www.projectliberty.org/}, a WS-Federation\footnote{http://docs.oasis-open.org/wsfed/federation/200706} e o \textit{framework} Shibboleth\footnote{http://shibboleth.net/}, que têm como foco respectivamente; ambientes corporativos e de \textit{e-commerce}, ambientes federados com foco na segurança, e sistema federado voltado para ambientes acadêmicos \cite{kallela:08}.
	
No Brasil, a \ac{RNP}, em conjunto com as instituições de ensino \acs{UFC}, \acs{UFMG}, \acs{UFF}, \acs{UFRGS} e \acs{CEFET-MG}, iniciaram o projeto da \ac{CAFe}\footnote{http://portal.rnp.br/web/servicos/cafe} com o intuito de reunir as universidades e instituições de pesquisa do País \cite{moreira:11}. Desde o ano de 2009, a \ac{RNP} disponibiliza o serviço da \ac{CAFe} às suas organizações usuárias. Através da \ac{CAFe}, um usuário mantém todas as suas informações na sua instituição de origem e pode acessar serviços oferecidos pelas instituições que participam da federação acadêmica.

\section{Problema de pesquisa na área de Gestão de Identidade}
\label{ci_s_problema}

Desenvolver pesquisa aplicada na área de gestão de identidades federadas exige que os experimentos sejam conduzidos em um ambiente que implemente uma federacão em sua totalidade, sendo que a complexidade de montar tal ambiente depende do \textit{framework} escolhido \cite{wangham:13}.
	
A federação \ac{CAFe} é um ambiente de produção, ou seja, nesta federação não deve ser permitida a realização de experimentos e assim pesquisadores que fazem prospecções tecnológicas e pesquisas científicas em gestão de identidade necessitam montar sua própria federação de testes para que possam conduzir seus projetos e experimentos.
	
Conceber uma federação acadêmica baseada no \textit{framework} Shibboleth para realizar experimentos práticos pode ser uma tarefa, muitas vezes, mais trabalhosa do que a implementação da pesquisa propriamente dita. Ter que implantar este ambiente complexo para o desenvolvimento da pesquisa, que demanda um tempo considerável dos pesquisadores envolvidos, para que então os experimentos possam ser executados pode inibir pesquisas na área. Outro complicador é o fato de ter que manter esse ambiente custoso, em termos de recursos computacionais, atualizações de segurança e de \textit{software}, entre outras atividades \cite{wangham:13}.

\section{Solução proposta}
\label{ci_s_proposta}

Ciente desta necessidade e com o intuito de motivar pesquisas em Gestão de Identidade, a RNP criou em 2013 o projeto, \ac{GId Lab}\footnote{http://wiki.rnp.br/display/gidlab/} que tem por objetivo geral disponibilizar para a comunidade acadêmica um ambiente virtual no qual os pesquisadores poderão realizar testes com \ac{IAA} e também \ac{ICP} \cite{wangham:13}.

Este trabalho tem como objetivo descrever o processo para implantar uma parcela do GId Lab, referente a IAA com o objetivo final de disponibilizar uma federação acadêmica para experimentação denominada, CAFe Expresso. A CAFe Expresso é constituída de Provedores de Identidade (IdP), Provedores de Serviço (SP) e dois diferentes serviço de descoberta, \ac{DS} um chamado \ac{WAYF}\footnote{https://wayf.switch.ch/} e outro chamado \ac{EDS}\footnote{http://shibboleth.net/products/embedded-discovery-service.html}, que realiza o redirecionamento do usuário para o seu IdP de origem para que este se autentique, sendo que o segundo, é uma implementação realizada pela própria equipe de desenvolvimento do projeto Shibboleth. Foi implementado também um serviço chamado \textit{uApprove}, que permite ao usuário saber quais atributos estão sendo liberados para o SP que deseja acessar, permitindo que o usuário aceite ou não a liberação destes atributos. Ainda no contexto deste trabalho, máquinas virtuais com IdP e SP serão configuradas e disponibilizadas para \textit{download} de forma a facilitar a implantação destes provedores nas instituições que estão realizando seus experimentos no GId Lab.

O presente trabalho foi desenvolvido dentro do escopo do projeto \ac{GId Lab}, sendo que o aluno é Bolsista de Iniciação Científica, financiado pela RNP.

\section{Objetivos}
\label{ci_s_objetivos}

O objetivo deste trabalho vai de encontro ao objetivo do próprio projeto GId Lab, que é disponibilizar para a comunidade científica um ambiente virtual para realização de pesquisas e testes em Gestão de Identidade em uma federação acadêmica para experimentação (CAFe Expresso) baseada no \textit{framework} Shibboleth. Além de disponibilizar o ambiente para acesso remoto, em servidores já configurados, será disponibilizado também duas possibilidades de ambientes através de máquinas virtuais pré-configuradas, um ambiente contém todos os elementes necessários para uma federação, composto por um IdP, um SP e um WAYF, para ser executado localmente na máquina do usuário. Outra possibilidade é o pesquisador interessado obter um dos elementos de uma federação baseada em Shibboleth, um IdP ou um SP (ou ainda ambos) possibilitando ao pesquisador possa configurar o provedor com as informações da sua instituição e realize teste através da CAFe Expresso, juntamente com outros pesquisadores.

\section{Estrutura do trabalho}
\label{ci_s_estrutura}

Este trabalho está dividido em cinco seções. O capítulo \ref{c_introducao}, contemplou uma introdução sobre os problemas atuais de gestão de identidades e as dificuldades enfrentadas por pesquisadores para implantação de um infraestrutura de gestão de identidades federadas. Em seguida os objetivos e a solução proposta foram descritos. O capítulo \ref{c_cap2} apresenta a fundamentação teórica necessária para compreensão dos conceitos, padrões e tecnologias envolvidas. O estudo bibliográfico realizado compreendeu os temas, gestão de identidades, o modelo de gerenciamento federado, a especificação SAML, que é base do \textit{framework} Shibboleth e, por fim, os componentes e funcionalidades do \textit{framework} Shibboleth. O capítulo \ref{c_cap3} descreve a federação CAFe, seus objetivos, assim como sua participação no cenário mundial, e lista alguns serviços disponíveis para usuários da federação CAFe. O capítulo \ref{c_cap4} apresenta uma visão geral da solução implantada, a CAFe Expresso, assim como as tecnologias e ferramentas (\textit{softwares}) utilizados para implantação e funciomento do \textit{framework} Shibboleth, além de mostrar uma avaliação sobre a solução proposta, por pesquisadores que fizeram uso e de usuários conhecedores de tecnologias de gestão de identidade federada, através de um questionário. Esta seção também aborda sucintamente a execução do projeto, citando os procedimentos realizados, para o completo funcionamento do trabalho proposto. Por fim o capítulo \ref{c_conclusao} apresenta a conclusão deste trabalho, expressando sobre a realização do trabalho proposto, como, motivação para implantação da infraestrutura e as possibilidades de trabalhos futuros.