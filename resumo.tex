% ----------------------------------------------------------------------- %
% Pequeno texto que em poucas palavras consegue expressar o trabalho.
% O resumo deve ser concebido de forma tal que, uma pessoa ao ler o resumo
% possa entender sobre qual assunto este trabalho trata.
%
% Arquivo: resumo.tex
% ----------------------------------------------------------------------- %

\begin{resumo}
A disponibilidade de serviços e aplicações acessíveis remotamente na Internet tornou a gestão de identidades uma estrutura complexa de manter, tanto para usuários quanto para administradores de sistemas. Para contornar isto, o modelo de gestão identidades federadas tem como objetivo facilitar o acesso aos serviços. No entanto a implantação de uma federação não é trivial, e, para muitos pesquisadores que estão desenvolvendo trabalhos nesta área, implantar uma federação para realizar experimentos práticos é uma tarefa custosa e demorada. Este trabalho tem como objetivo implantar e disponibilizar uma infraestrutura para que pesquisadores possam conduzir experimentos em uma federação acadêmica baseada no framework Shibboleth. Dos objetivos propostos todos foram executados, resultando em uma federação acadêmica para experimentação utilizando o \textit{framework} Shibboleth, que permite testes e desenvolvimento de tecnologias no âmbito de gestão de identidade federada. Os resultados são máquinas virtuais pré-configuradas com uma federação completa que pode ser executadas localmente ou elementos de uma federação para adesão na federação para experimentação desenvolvida neste trabalho.
\end{resumo}
