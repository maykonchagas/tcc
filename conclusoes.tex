% ----------------------------------------------------------------------- %
% Arquivo: conclusoes.tex
% ----------------------------------------------------------------------- %
\chapter{Conclusões}
\label{c_conclusao}

O crescimento da internet, como meio para realização de negócios, estudos, trocas de informações de forma geral, entre Homem X Máquina (\textit{Human to Machine} -- H2M) e Máquina X Máquina (\textit{Machine to Machine} -- M2M), criou uma necessidade natural de identificação das entidades que convivem na rede de computadores. Uma necessidade que é tratada no processo de gestão de identidades, por meio de divermos modelos.

O conceito de Gestão de Identidades Federada têm crescido nos últimos anos e refere-se a um conjunto de tecnologias e padrões que permite a interação do usuário com diversos serviços usando apenas uma credencial de acesso. A função básica da identidade federada, o SSO, possibilita ao usuário o uso da autenticação feita em um \textit{site} e o uso desta mesma validação para acessar outros serviços protegidos \cite{kallela:08}.

No trabalho em questão, foi abordado o modelo de gestão de identidades federadas, como implementado no \textit{framework} Shibboleth. Este modelo permite a descentralização dos provedores de identidade dos provedores de serviço, que além de facilitar o gerenciamento da infraestrutura dos provedores, para os administradores destes, também facilita para o usuário que precisará de somente uma única identificação para acesso aos serviços disponibilizados pelos provedores de serviço. No modelo de gestão de identidades federadas a especificação mais utilizada é o SAML, que define que tipos de informações são trocadas pelos provedores de identidades e de  serviços. O \textit{framework} Shibboleth é o mais utilizado em ambientes acadêmicos.

Gestão de Identidades federadas é uma área ativa de pesquisa, sendo que muitos trabalhos desenvolvidos nesta área precisam realizar experimentos com soluções e frameworks consolidados como o Shibboleth. Desenvolver pesquisas aplicadas na área de gestão de identidades federadas exige que os experimentos sejam conduzidos em um ambiente que implemente uma federação em sua totalidade. Configurar uma federação  para realizar experimentos de uma pesquisa, pode ser uma tarefa mais árdua e demorada do que a implementação da pesquisa propriamente dita \cite{wangham:13}.

O objetivo deste trabalho era implantar uma parte do GId Lab, um ambiente virtual de apoio aos pesquisadores brasileiros a fim de estimular e facilitar o desenvolvimento de novas soluções que possam vir a ser disponibilizadas na federação acadêmica, CAFe, ou como um serviço da RNP. Do objetivo proposto no início do trabalho, todas as atividades foram realizadas. De uma forma geral, a implantação da federação acadêmica foi realizada em sua plenitude, o ambiente é composto de três Provedores de Identidade \ac{IdP}, três Provedores de Serviço \ac{SP}, dois serviços de descoberta, o \ac{WAYF} e o \ac{EDS} e um serviço que solicita o consentimento do usuário para liberação dos atributos solicitados pelo \ac{SP} ao IdP, o \textit{uApprove}. Além disto, foram disponibilizadas máquinas virtuais para \textit{download} por pesquisadores interessados em implantar uma federação Shibboleth. Duas categorias de máquinas virtuais foram disponibilizadas, é possível realizar \textit{download} dos elementos Shibboleth, separadamente, para fazer testes na CAFe Expresso, ou uma federação completa para uso local.

Para trabalhos futuros podem ser sugeridos a implantação do IdP+, que é um IdP para tradução de credenciais de segurança, permitindo a geração de certificados X.509 e permitindo que aplicações não \textit{web} possam fazer uso de autenticação federadas Shibboleth. Outra sugestão para trabalhos futuros é a implantação do \ac{SGC} que permite a tradução de credenciais Shibboleth em certificados digitais, que podem ser consumidas por serviços que requerem estes tipos de certificados. Mais uma sugestão de trabalho futuro é a implementação do \ac{SLO}. Atualmente a forma de se deslogar de uma sessão federada, é fechar o navegador \textit{web}, com o \ac{SLO} seria possível finalizar a sessão com um único clique. No entanto esta funcionalidade não é totalmente suportada pelo \textit{framework} Shibboleth, e só será na próxima versão, que não tem previsão de lançamento oficial. Uma última sugestão de trabalhos futuros é integração entre a federação CAFe Expresso, que utiliza a especificação SAML através do \textit{framework} Shibboleth e outras tecnologias de gestão de identidade federada, como OAuth\footnote{http://oauth.net/} e OpenID Connect\footnote{http://openid.net/} que implementam outros padrões de comunicação, diferentes do SAML.