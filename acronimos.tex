%%%%%%%%%%%%%% Como usar o pacote acronym
% \ac{acronimo} -- Na primeira vez que for citado o acronimo, o nome completo irá aparecer
%                  seguido do acronimo entre parênteses. Na proxima vez somente o acronimo
%                  irá aparecer. Se usou a opção footnote no pacote, entao o nome por extenso
%                  irá aparecer aparecer no rodapé
%
% \acf{acronimo} -- Para aparecer com nome completo + acronimo
% \acs{acronimo} -- Para aparecer somente o acronimo
% \acl{acronimo} -- Nome por extenso somente, sem o acronimo
% \acp{acronimo} -- igual o \ac mas deixando no plural com S (ingles)
% \acfp{acronimo}--
% \acsp{acronimo}--
% \aclp{acronimo}--
%%%%%%%% ATENCAO
% Criei o comando \acfe{}, resultando em: Extenso -- ACRO

\chapter*{Lista de Abreviaturas}%
% \addcontentsline{toc}{chapter}{Lista de abreviaturas}
\markboth{Lista de abreviaturas}{}


\begin{acronym}
	\acro{IdP}{\textit{Identity Provider}}	
	\acro{SP}{\textit{Service Provider}}
	\acro{SSO}{\textit{Single Sign-On}}
	\acro{RNP}{Rede Nacional de Ensino e Pesquisa}
	\acro{UFC}{Universidade Federal do Ceará}
	\acro{UFMG}{Universidade Federal de Minas Gerais}
	\acro{UFF}{Universidade Federal Fluminense}
	\acro{UFRGS}{Universidade Federal do Rio Grande do Sul}
	\acro{CEFET-MG}{Centro Federal de Educação Tecnológica de Minas Gerais}
	\acro{CAFe}{Comunidade Acadêmica Federada}
	\acro{GId Lab}{Laboratório de Experimentação em Gestão de Identidades}
	\acro{IAA}{Infraestrutura de Autenticação e Autorização}
	\acro{ICP}{Infraestrutura de Chave Pública}
	\acro{WAYF}{\textit{Where Are You From}}
	\acro{DS}{\textit{Discovery Service}}
	\acro{EDS}{\textit{Embedded Discovery Service}}
	\acro{SAML}{\textit{Security Assertion Markup Language}}
	\acro{URL}{\textit{Uniform Resource Locator}}
	\acro{LDAP}{\textit{Lightweight Directory Access Protocol}}
	\acro{CA}{\textit{Certificate Authority}}
	\acro{XML}{\textit{Extensible Markup Language}}
	\acro{SSTC}{\textit{Security Services Technical Committee}}
	\acro{OASIS}{\textit{Organization for the Advancement of Structured Information Standard}}
	\acro{SOAP}{\textit{Simple Object Access Protocol}}
	\acro{MACE}{\textit{Middleware Architecture Committee for Education}}
	\acro{SCHAC}{\textit{SCHema for ACademia}}
	\acro{VoIP}{\textit{Voice over IP}}
	\acro{TERENA}{\textit{Trans-European Research and Education Networking}}
	\acro{NREN}{\textit{National Research and Education Network}}
	\acro{PoP}{Ponto de Presença}
	\acro{CSS}{\textit{Cascading Style Sheets}}
	\acro{NTP}{\textit{Network Time Protocol}}
	\acro{PGID}{Programa de Gestão de Identidade}
	\acro{GT}{Grupo de Trabalho}
	\acro{SGCI}{Sistema de Gerenciamento de Certificados Digitais ICPEdu}
	\acro{ICPEdu}{Infraestrutura de Chaves Públicas para Ensino e Pesquisa}
	\acro{AC}{Autoridade Certificadora}
	\acro{GId}{Gestão de Identidade}
	\acro{VM}{\textit{Virtual Machine}}
	\acro{GT-STCFed}{Grupo de Trabalho Serviços para Transposição de Credenciais de Autenticação Federadas}
	\acro{CT-GId}{Comitê Técnico de Gestão de Identidade}
	\acro{SSL}{\textit{Secure Software Layer}}
	\acro{TLS}{\textit{Transport Layer Security}}
	\acro{SSH}{\textit{Secure Shell}}
	\acro{SGC}{Serviço Gerador de Certificados}
	\acro{SLO}{\textit{Single Logout}}
\end{acronym}